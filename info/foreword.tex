\begin{center}
KATA SAMBUTAN\\
PROGRAM STUDI SARJANA TERAPAN TEKNIK INFORMATIKA\\
UNIVERSITAS LOGISTIK DAN BISNIS INTERNASIONAL (ULBI)\\[1cm]
\end{center}

Shalom, Om Swastiastu, Namo Budaya, Salam Kebajikan. Salam sejahtera untuk kita semua.\\

Teknologi Informasi merupakan teknologi yang mendominasi berbagai aspek kehidupan pada saat ini. Perkembangan teknologi informasi berjalan dengan sangat cepatnya. Banyak industri yang menggunakan teknologi informasi sebagai sarana pendukung untuk meningkatkan kinerjanya. Instansi pemerintah, swasta, dan berbagai sektor lainnya juga tidak terlepas dari peran teknologi informasi sebagai pendukung operasionalnya.\\

Kebutuhan terhadap tenaga terampil dibidang teknologi informasi ini sangat banyak diperlukan. Menyadari akan hal tersebut, maka dirilislah program Sarjana TerapanTeknik Informatika yang penyelenggaraannya dimulai pada bulan Oktober 2011 berdasarkan Surat Keputusan No 241/E/O/2011 tanggal 17 Oktober 2011 tentang pendirian program studi. Sesuai dengan core business PT. Pos Indonesia, program studi ini menspesifikasikan dirinya di bidang pengembangan Teknologi Informasi dan Komunikasi yang mendukung bidang Logistik dan Manajemen Rantai Pasok sebagai kekhasan yang dibentuk untuk membedakan dengan perguruan tinggi lain yang menyelenggarakan program studi yang sama.\\

Program Studi Sarjana Terapan Teknik Informatika saat ini merupakan salah satu program studi di lingkungan Fakultas Sekolah Vokasi Universitas Logistik dan Bisnis Internasional (ULBI) yang berkomitmen menghasilkan sarjana terapan teknik informatika yang professional, berkarakter dan tentunya mampu bersaing di dunia kerja, komitmen ini ditunjukan melalui penerapan kurikulum yang sesuai dengan standar SN-DIKTI dan Kebutuhan industri.\\

Profil lulusan yang dikembangkan oleh Program Studi Sarjana Terapan Teknik Informatika adalah sebagai individu yang mampu memberikan solusi dan keputusan strategis terutama pada bidang System Analyst, Computer Support Specialist, Database Administrator, Programmer dan Software Developer.\\

Publikasi buku popular Ini adalah salah satu upaya mensinergikan visi misi program studi untuk menjadi program studi yang unggul secara nasional dibidang teknologi informasi yang mendukung Bidang Logistik dan Manajemen Rantai Pasok, Menghasilkan tenaga professional dibidang teknologi informasi dan Komunikasi (TIK), Menerapkan ilmu pengetahuan dan TIK yang relevan dengan peningkatan layanan TIK di industri logistik serta melaksanakan pengabdian kepada masyarakat untuk memenuhi kebutuhan industrialisasi dan meningkatkan kesejahteraan masyarakat dengan pembekalan ilmu pengetahuan dan teknologi.\\

Tetap semangat dalam berkarya, bangun reputasi dengan prestasi, semoga Alloh SWT selalu memberikan keberkahan untuk kita semua. Aamiin…\\
Wassalamualaikum, Wr. Wb

\begin{flushleft}
Bandung, Agustus 2022\\
Ketua Program Studi\\
Roni Andarsyah, ST., M.Kom., SFPC\\[1cm]
\end{flushleft}